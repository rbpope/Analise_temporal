% Options for packages loaded elsewhere
\PassOptionsToPackage{unicode}{hyperref}
\PassOptionsToPackage{hyphens}{url}
%
\documentclass[
]{article}
\usepackage{amsmath,amssymb}
\usepackage{lmodern}
\usepackage{ifxetex,ifluatex}
\ifnum 0\ifxetex 1\fi\ifluatex 1\fi=0 % if pdftex
  \usepackage[T1]{fontenc}
  \usepackage[utf8]{inputenc}
  \usepackage{textcomp} % provide euro and other symbols
\else % if luatex or xetex
  \usepackage{unicode-math}
  \defaultfontfeatures{Scale=MatchLowercase}
  \defaultfontfeatures[\rmfamily]{Ligatures=TeX,Scale=1}
\fi
% Use upquote if available, for straight quotes in verbatim environments
\IfFileExists{upquote.sty}{\usepackage{upquote}}{}
\IfFileExists{microtype.sty}{% use microtype if available
  \usepackage[]{microtype}
  \UseMicrotypeSet[protrusion]{basicmath} % disable protrusion for tt fonts
}{}
\makeatletter
\@ifundefined{KOMAClassName}{% if non-KOMA class
  \IfFileExists{parskip.sty}{%
    \usepackage{parskip}
  }{% else
    \setlength{\parindent}{0pt}
    \setlength{\parskip}{6pt plus 2pt minus 1pt}}
}{% if KOMA class
  \KOMAoptions{parskip=half}}
\makeatother
\usepackage{xcolor}
\IfFileExists{xurl.sty}{\usepackage{xurl}}{} % add URL line breaks if available
\IfFileExists{bookmark.sty}{\usepackage{bookmark}}{\usepackage{hyperref}}
\hypersetup{
  pdftitle={Trabalho 1 de Análise de séries temporais},
  hidelinks,
  pdfcreator={LaTeX via pandoc}}
\urlstyle{same} % disable monospaced font for URLs
\usepackage[margin=1in]{geometry}
\usepackage{color}
\usepackage{fancyvrb}
\newcommand{\VerbBar}{|}
\newcommand{\VERB}{\Verb[commandchars=\\\{\}]}
\DefineVerbatimEnvironment{Highlighting}{Verbatim}{commandchars=\\\{\}}
% Add ',fontsize=\small' for more characters per line
\usepackage{framed}
\definecolor{shadecolor}{RGB}{248,248,248}
\newenvironment{Shaded}{\begin{snugshade}}{\end{snugshade}}
\newcommand{\AlertTok}[1]{\textcolor[rgb]{0.94,0.16,0.16}{#1}}
\newcommand{\AnnotationTok}[1]{\textcolor[rgb]{0.56,0.35,0.01}{\textbf{\textit{#1}}}}
\newcommand{\AttributeTok}[1]{\textcolor[rgb]{0.77,0.63,0.00}{#1}}
\newcommand{\BaseNTok}[1]{\textcolor[rgb]{0.00,0.00,0.81}{#1}}
\newcommand{\BuiltInTok}[1]{#1}
\newcommand{\CharTok}[1]{\textcolor[rgb]{0.31,0.60,0.02}{#1}}
\newcommand{\CommentTok}[1]{\textcolor[rgb]{0.56,0.35,0.01}{\textit{#1}}}
\newcommand{\CommentVarTok}[1]{\textcolor[rgb]{0.56,0.35,0.01}{\textbf{\textit{#1}}}}
\newcommand{\ConstantTok}[1]{\textcolor[rgb]{0.00,0.00,0.00}{#1}}
\newcommand{\ControlFlowTok}[1]{\textcolor[rgb]{0.13,0.29,0.53}{\textbf{#1}}}
\newcommand{\DataTypeTok}[1]{\textcolor[rgb]{0.13,0.29,0.53}{#1}}
\newcommand{\DecValTok}[1]{\textcolor[rgb]{0.00,0.00,0.81}{#1}}
\newcommand{\DocumentationTok}[1]{\textcolor[rgb]{0.56,0.35,0.01}{\textbf{\textit{#1}}}}
\newcommand{\ErrorTok}[1]{\textcolor[rgb]{0.64,0.00,0.00}{\textbf{#1}}}
\newcommand{\ExtensionTok}[1]{#1}
\newcommand{\FloatTok}[1]{\textcolor[rgb]{0.00,0.00,0.81}{#1}}
\newcommand{\FunctionTok}[1]{\textcolor[rgb]{0.00,0.00,0.00}{#1}}
\newcommand{\ImportTok}[1]{#1}
\newcommand{\InformationTok}[1]{\textcolor[rgb]{0.56,0.35,0.01}{\textbf{\textit{#1}}}}
\newcommand{\KeywordTok}[1]{\textcolor[rgb]{0.13,0.29,0.53}{\textbf{#1}}}
\newcommand{\NormalTok}[1]{#1}
\newcommand{\OperatorTok}[1]{\textcolor[rgb]{0.81,0.36,0.00}{\textbf{#1}}}
\newcommand{\OtherTok}[1]{\textcolor[rgb]{0.56,0.35,0.01}{#1}}
\newcommand{\PreprocessorTok}[1]{\textcolor[rgb]{0.56,0.35,0.01}{\textit{#1}}}
\newcommand{\RegionMarkerTok}[1]{#1}
\newcommand{\SpecialCharTok}[1]{\textcolor[rgb]{0.00,0.00,0.00}{#1}}
\newcommand{\SpecialStringTok}[1]{\textcolor[rgb]{0.31,0.60,0.02}{#1}}
\newcommand{\StringTok}[1]{\textcolor[rgb]{0.31,0.60,0.02}{#1}}
\newcommand{\VariableTok}[1]{\textcolor[rgb]{0.00,0.00,0.00}{#1}}
\newcommand{\VerbatimStringTok}[1]{\textcolor[rgb]{0.31,0.60,0.02}{#1}}
\newcommand{\WarningTok}[1]{\textcolor[rgb]{0.56,0.35,0.01}{\textbf{\textit{#1}}}}
\usepackage{graphicx}
\makeatletter
\def\maxwidth{\ifdim\Gin@nat@width>\linewidth\linewidth\else\Gin@nat@width\fi}
\def\maxheight{\ifdim\Gin@nat@height>\textheight\textheight\else\Gin@nat@height\fi}
\makeatother
% Scale images if necessary, so that they will not overflow the page
% margins by default, and it is still possible to overwrite the defaults
% using explicit options in \includegraphics[width, height, ...]{}
\setkeys{Gin}{width=\maxwidth,height=\maxheight,keepaspectratio}
% Set default figure placement to htbp
\makeatletter
\def\fps@figure{htbp}
\makeatother
\setlength{\emergencystretch}{3em} % prevent overfull lines
\providecommand{\tightlist}{%
  \setlength{\itemsep}{0pt}\setlength{\parskip}{0pt}}
\setcounter{secnumdepth}{-\maxdimen} % remove section numbering
\ifluatex
  \usepackage{selnolig}  % disable illegal ligatures
\fi

\title{Trabalho 1 de Análise de séries temporais}
\author{}
\date{\vspace{-2.5em}}

\begin{document}
\maketitle

O Objetivo deste trabalho é de responder as seguintes arguições:

\begin{enumerate}
\def\labelenumi{\arabic{enumi}.}
\item
  Carregar o arquivo: DepartmentStoreSales\_V2.xls
\item
  Plotar o gráfico da série temporal
\item
  Calcular o resumo estatístico da base
\item
  Separar em amostra de desenvolvimento e teste da seguinte maneira 4.1
  Desenvolvimento: 2005/1 até 2009/4 4.2 Teste: 2010/1 até 2010/4
\item
  Estimar um modelo de tendência linear
\item
  Estimar um modelo de tendência quadrática (polinômio de grau 2)
\item
  Estimar um modelo de tendência linear com sazonalidade
\item
  Estimar um modelo de tendência quadrática (polinômio de grau 2) com
  sazonalidade
\item
  Calcular os erros de projeção para cada modelo
\item
  Escolher o melhor modelo de projeção justificando
\item
  Carregar o arquivo: DepartmentStoreSalesV2.xls
\end{enumerate}

\begin{Shaded}
\begin{Highlighting}[]
\NormalTok{sales}\OtherTok{\textless{}{-}} \FunctionTok{read\_excel}\NormalTok{(}\StringTok{"DepartmentStoreSales\_V2.xls"}\NormalTok{)}
\end{Highlighting}
\end{Shaded}

\begin{enumerate}
\def\labelenumi{\arabic{enumi}.}
\setcounter{enumi}{1}
\tightlist
\item
  Plotar o gráfico da série temporal: Converter em série temporal
\end{enumerate}

\begin{Shaded}
\begin{Highlighting}[]
\NormalTok{sales\_ts }\OtherTok{\textless{}{-}} \FunctionTok{ts}\NormalTok{(sales}\SpecialCharTok{$}\NormalTok{Sales, }\AttributeTok{start=}\FunctionTok{c}\NormalTok{(}\DecValTok{2005}\NormalTok{,}\DecValTok{1}\NormalTok{), }\AttributeTok{end=}\FunctionTok{c}\NormalTok{(}\DecValTok{2010}\NormalTok{,}\DecValTok{04}\NormalTok{), }\AttributeTok{frequency =} \DecValTok{4}\NormalTok{)}
\end{Highlighting}
\end{Shaded}

Plotar os gráficos:

\begin{Shaded}
\begin{Highlighting}[]
\FunctionTok{plot}\NormalTok{(sales\_ts, }\AttributeTok{xlab=}\StringTok{"Tempo"}\NormalTok{, }\AttributeTok{ylab=}\StringTok{"Vendas"}\NormalTok{,}\AttributeTok{ylim=}\FunctionTok{c}\NormalTok{(}\DecValTok{40000}\NormalTok{, }\DecValTok{110000}\NormalTok{), }\AttributeTok{type =} \StringTok{"l"}\NormalTok{)}
\end{Highlighting}
\end{Shaded}

\includegraphics{Trabalho_files/figure-latex/unnamed-chunk-2-1.pdf}

\begin{Shaded}
\begin{Highlighting}[]
\FunctionTok{plot}\NormalTok{(sales}\SpecialCharTok{$}\NormalTok{Sales, }\AttributeTok{xlab=}\StringTok{"Tempo"}\NormalTok{, }\AttributeTok{ylab=}\StringTok{"Vendas"}\NormalTok{, }\AttributeTok{ylim=}\FunctionTok{c}\NormalTok{(}\DecValTok{40000}\NormalTok{, }\DecValTok{110000}\NormalTok{), }\AttributeTok{type=}\StringTok{"p"}\NormalTok{)}
\end{Highlighting}
\end{Shaded}

\includegraphics{Trabalho_files/figure-latex/unnamed-chunk-2-2.pdf} 3.
Calcular Resumo estatístico da base:

\begin{Shaded}
\begin{Highlighting}[]
\FunctionTok{summary}\NormalTok{(sales)}
\end{Highlighting}
\end{Shaded}

\begin{verbatim}
##       Year         Quarter          Sales       
##  Min.   :2005   Min.   : 1.00   Min.   : 48617  
##  1st Qu.:2006   1st Qu.: 6.75   1st Qu.: 52681  
##  Median :2008   Median :12.50   Median : 59440  
##  Mean   :2008   Mean   :12.50   Mean   : 64757  
##  3rd Qu.:2009   3rd Qu.:18.25   3rd Qu.: 76835  
##  Max.   :2010   Max.   :24.00   Max.   :103337
\end{verbatim}

\begin{Shaded}
\begin{Highlighting}[]
\FunctionTok{summary}\NormalTok{(sales\_ts)}
\end{Highlighting}
\end{Shaded}

\begin{verbatim}
##    Min. 1st Qu.  Median    Mean 3rd Qu.    Max. 
##   48617   52681   59440   64757   76835  103337
\end{verbatim}

\begin{enumerate}
\def\labelenumi{\arabic{enumi}.}
\setcounter{enumi}{3}
\tightlist
\item
  Separar em amostra de desenvolvimento e teste da seguinte maneira 4.1
  Desenvolvimento: 2005/1 até 2009/4
\end{enumerate}

\begin{Shaded}
\begin{Highlighting}[]
\NormalTok{tam\_amostra\_teste }\OtherTok{\textless{}{-}} \DecValTok{4}
\NormalTok{tam\_amostra\_treinamento }\OtherTok{\textless{}{-}} \FunctionTok{length}\NormalTok{(sales\_ts) }\SpecialCharTok{{-}}\NormalTok{ tam\_amostra\_teste}
\NormalTok{treinamento\_ts }\OtherTok{\textless{}{-}} \FunctionTok{window}\NormalTok{(sales\_ts, }\AttributeTok{start=}\FunctionTok{c}\NormalTok{(}\DecValTok{2005}\NormalTok{, }\DecValTok{1}\NormalTok{), }\AttributeTok{end=}\FunctionTok{c}\NormalTok{(}\DecValTok{2005}\NormalTok{,tam\_amostra\_treinamento))}
\NormalTok{treinamento\_ts}
\end{Highlighting}
\end{Shaded}

\begin{verbatim}
##       Qtr1  Qtr2  Qtr3  Qtr4
## 2005 50147 49325 57048 76781
## 2006 48617 50898 58517 77691
## 2007 50862 53028 58849 79660
## 2008 51640 54119 65681 85175
## 2009 56405 60031 71486 92183
\end{verbatim}

4.2 Teste: 2010/1 até 2010/4

\begin{Shaded}
\begin{Highlighting}[]
\NormalTok{validacao\_ts }\OtherTok{\textless{}{-}} \FunctionTok{window}\NormalTok{(sales\_ts, }\AttributeTok{start=}\FunctionTok{c}\NormalTok{(}\DecValTok{2005}\NormalTok{, tam\_amostra\_treinamento }\SpecialCharTok{+} \DecValTok{1}\NormalTok{), }\AttributeTok{end=}\FunctionTok{c}\NormalTok{(}\DecValTok{2005}\NormalTok{,tam\_amostra\_treinamento}\SpecialCharTok{+}\NormalTok{tam\_amostra\_teste))}
\NormalTok{validacao\_ts}
\end{Highlighting}
\end{Shaded}

\begin{verbatim}
##        Qtr1   Qtr2   Qtr3   Qtr4
## 2010  60800  64900  76997 103337
\end{verbatim}

Avaliação gráfica do treinamento e validação

\begin{Shaded}
\begin{Highlighting}[]
\FunctionTok{plot}\NormalTok{(treinamento\_ts, }\AttributeTok{xlab=}\StringTok{"Tempo"}\NormalTok{, }\AttributeTok{ylab=}\StringTok{"Vendas"}\NormalTok{, }\AttributeTok{xaxt=}\StringTok{"n"}\NormalTok{ , }\AttributeTok{ylim=}\FunctionTok{c}\NormalTok{(}\DecValTok{40000}\NormalTok{, }\DecValTok{110000}\NormalTok{), }\AttributeTok{xlim=}\FunctionTok{c}\NormalTok{(}\DecValTok{2005}\NormalTok{, }\FloatTok{2011.25}\NormalTok{), }\AttributeTok{bty=}\StringTok{"l"}\NormalTok{)}

\FunctionTok{axis}\NormalTok{(}\DecValTok{1}\NormalTok{, }\AttributeTok{at=}\FunctionTok{seq}\NormalTok{(}\DecValTok{2005}\NormalTok{, }\DecValTok{2010}\NormalTok{, }\DecValTok{24}\NormalTok{), }\AttributeTok{labels=}\FunctionTok{format}\NormalTok{(}\FunctionTok{seq}\NormalTok{(}\DecValTok{2005}\NormalTok{, }\DecValTok{2010}\NormalTok{,}\DecValTok{24}\NormalTok{)))}

\FunctionTok{lines}\NormalTok{(validacao\_ts, }\AttributeTok{bty=}\StringTok{"l"}\NormalTok{, }\AttributeTok{col=}\StringTok{"red"}\NormalTok{)}
\end{Highlighting}
\end{Shaded}

\includegraphics{Trabalho_files/figure-latex/unnamed-chunk-6-1.pdf}

\begin{enumerate}
\def\labelenumi{\arabic{enumi}.}
\setcounter{enumi}{4}
\tightlist
\item
  Estimar um modelo de tendência linear
\end{enumerate}

\begin{Shaded}
\begin{Highlighting}[]
\NormalTok{modelo\_tendencia\_linear }\OtherTok{\textless{}{-}} \FunctionTok{tslm}\NormalTok{(treinamento\_ts }\SpecialCharTok{\textasciitilde{}}\NormalTok{ trend)}
\FunctionTok{summary}\NormalTok{(modelo\_tendencia\_linear)}
\end{Highlighting}
\end{Shaded}

\begin{verbatim}
## 
## Call:
## tslm(formula = treinamento_ts ~ trend)
## 
## Residuals:
##    Min     1Q Median     3Q    Max 
## -13439  -9119  -3051   5906  21322 
## 
## Coefficients:
##             Estimate Std. Error t value Pr(>|t|)    
## (Intercept)  51183.7     5609.0   9.125 3.58e-08 ***
## trend         1068.9      468.2   2.283   0.0348 *  
## ---
## Signif. codes:  0 '***' 0.001 '**' 0.01 '*' 0.05 '.' 0.1 ' ' 1
## 
## Residual standard error: 12070 on 18 degrees of freedom
## Multiple R-squared:  0.2245, Adjusted R-squared:  0.1814 
## F-statistic: 5.211 on 1 and 18 DF,  p-value: 0.03482
\end{verbatim}

Plotar resíduos

\begin{Shaded}
\begin{Highlighting}[]
\FunctionTok{plot}\NormalTok{(modelo\_tendencia\_linear}\SpecialCharTok{$}\NormalTok{residuals, }\AttributeTok{xlab=}\StringTok{"Tempo"}\NormalTok{, }\AttributeTok{ylab=}\StringTok{"Resíduos"}\NormalTok{, }\AttributeTok{ylim=}\FunctionTok{c}\NormalTok{(}\SpecialCharTok{{-}}\DecValTok{50000}\NormalTok{, }\DecValTok{50000}\NormalTok{), }\AttributeTok{bty=}\StringTok{"l"}\NormalTok{)}
\end{Highlighting}
\end{Shaded}

\includegraphics{Trabalho_files/figure-latex/unnamed-chunk-8-1.pdf}
Calcular a auto-correlação dos resíduos

\begin{Shaded}
\begin{Highlighting}[]
\FunctionTok{Acf}\NormalTok{(modelo\_tendencia\_linear}\SpecialCharTok{$}\NormalTok{residuals, }\AttributeTok{main=}\StringTok{"Auto{-}correlação dos Resíduos"}\NormalTok{)}
\end{Highlighting}
\end{Shaded}

\includegraphics{Trabalho_files/figure-latex/unnamed-chunk-9-1.pdf}
Verificar resíduos com teste de ``Ljung-Box''

\begin{Shaded}
\begin{Highlighting}[]
\FunctionTok{checkresiduals}\NormalTok{(modelo\_tendencia\_linear, }\AttributeTok{test=}\StringTok{"LB"}\NormalTok{)}
\end{Highlighting}
\end{Shaded}

\includegraphics{Trabalho_files/figure-latex/unnamed-chunk-10-1.pdf}

\begin{verbatim}
## 
##  Ljung-Box test
## 
## data:  Residuals from Linear regression model
## Q* = 23.539, df = 3, p-value = 3.117e-05
## 
## Model df: 2.   Total lags used: 5
\end{verbatim}

\begin{enumerate}
\def\labelenumi{\arabic{enumi}.}
\setcounter{enumi}{5}
\tightlist
\item
  Estimar um modelo de tendência quadrática (polinômio de grau 2)
\end{enumerate}

\begin{Shaded}
\begin{Highlighting}[]
\NormalTok{modelo\_tendencia\_poli }\OtherTok{\textless{}{-}} \FunctionTok{tslm}\NormalTok{(treinamento\_ts }\SpecialCharTok{\textasciitilde{}}\NormalTok{ trend }\SpecialCharTok{+} \FunctionTok{I}\NormalTok{(trend}\SpecialCharTok{\^{}}\DecValTok{2}\NormalTok{))}
\NormalTok{modelo\_tendencia\_poli}
\end{Highlighting}
\end{Shaded}

\begin{verbatim}
## 
## Call:
## tslm(formula = treinamento_ts ~ trend + I(trend^2))
## 
## Coefficients:
## (Intercept)        trend   I(trend^2)  
##    56462.70      -370.83        68.56
\end{verbatim}

Plotar resíduos

\begin{Shaded}
\begin{Highlighting}[]
\FunctionTok{plot}\NormalTok{(modelo\_tendencia\_poli}\SpecialCharTok{$}\NormalTok{residuals, }\AttributeTok{xlab=}\StringTok{"Tempo"}\NormalTok{, }\AttributeTok{ylab=}\StringTok{"Resíduos"}\NormalTok{, }\AttributeTok{bty=}\StringTok{"l"}\NormalTok{)}
\end{Highlighting}
\end{Shaded}

\includegraphics{Trabalho_files/figure-latex/unnamed-chunk-12-1.pdf}
Calcular a auto-correlação dos resíduos

\begin{Shaded}
\begin{Highlighting}[]
\FunctionTok{Acf}\NormalTok{(modelo\_tendencia\_poli}\SpecialCharTok{$}\NormalTok{residuals, }\AttributeTok{main=}\StringTok{"Auto{-}correlação dos Resíduos"}\NormalTok{)}
\end{Highlighting}
\end{Shaded}

\includegraphics{Trabalho_files/figure-latex/unnamed-chunk-13-1.pdf}
Verificar resíduos com teste de ``Ljung-Box''

\begin{Shaded}
\begin{Highlighting}[]
\FunctionTok{checkresiduals}\NormalTok{(modelo\_tendencia\_poli}\SpecialCharTok{$}\NormalTok{residuals, }\AttributeTok{test=}\StringTok{"LB"}\NormalTok{)}
\end{Highlighting}
\end{Shaded}

\begin{verbatim}
## Warning in modeldf.default(object): Could not find appropriate degrees of
## freedom for this model.
\end{verbatim}

\includegraphics{Trabalho_files/figure-latex/unnamed-chunk-14-1.pdf}
plot do modelo com tendência

\begin{Shaded}
\begin{Highlighting}[]
\FunctionTok{plot}\NormalTok{(treinamento\_ts, }\AttributeTok{xlab=}\StringTok{"Tempo"}\NormalTok{, }\AttributeTok{ylab=}\StringTok{"Vendas"}\NormalTok{, }\AttributeTok{bty=}\StringTok{"l"}\NormalTok{)}
\FunctionTok{lines}\NormalTok{(modelo\_tendencia\_poli}\SpecialCharTok{$}\NormalTok{fitted.values, }\AttributeTok{lwd=}\DecValTok{2}\NormalTok{)}
\end{Highlighting}
\end{Shaded}

\includegraphics{Trabalho_files/figure-latex/unnamed-chunk-15-1.pdf}
Projeta o modelo no período de Validação

\begin{Shaded}
\begin{Highlighting}[]
\NormalTok{modelo\_tendencia\_poli\_proj }\OtherTok{\textless{}{-}} \FunctionTok{forecast}\NormalTok{(modelo\_tendencia\_poli, }\AttributeTok{h =}\NormalTok{ tam\_amostra\_teste, }\AttributeTok{level=}\FloatTok{0.95}\NormalTok{)}
\FunctionTok{plot}\NormalTok{(modelo\_tendencia\_poli\_proj, }\AttributeTok{xlab=}\StringTok{"Tempo"}\NormalTok{, }\AttributeTok{ylab=}\StringTok{"Vendas"}\NormalTok{, }\AttributeTok{xaxt=}\StringTok{"n"}\NormalTok{ , }\AttributeTok{ylim=}\FunctionTok{c}\NormalTok{(}\DecValTok{40000}\NormalTok{, }\DecValTok{110000}\NormalTok{), }\AttributeTok{xlim=}\FunctionTok{c}\NormalTok{(}\DecValTok{2005}\NormalTok{, }\FloatTok{2010.25}\NormalTok{), }\AttributeTok{bty=}\StringTok{"l"}\NormalTok{, }\AttributeTok{flty=}\DecValTok{2}\NormalTok{,}\AttributeTok{main=}\StringTok{"Forecast from Polynomial regression model"}\NormalTok{)}

\FunctionTok{axis}\NormalTok{(}\DecValTok{1}\NormalTok{, }\AttributeTok{at=}\FunctionTok{seq}\NormalTok{(}\DecValTok{2005}\NormalTok{, }\DecValTok{2010}\NormalTok{, }\DecValTok{1}\NormalTok{), }\AttributeTok{labels=}\FunctionTok{format}\NormalTok{(}\FunctionTok{seq}\NormalTok{(}\DecValTok{2005}\NormalTok{, }\DecValTok{2010}\NormalTok{,}\DecValTok{1}\NormalTok{)))}

\FunctionTok{lines}\NormalTok{(validacao\_ts)}
\FunctionTok{lines}\NormalTok{(modelo\_tendencia\_poli\_proj}\SpecialCharTok{$}\NormalTok{fitted, }\AttributeTok{lwd=}\DecValTok{2}\NormalTok{, }\AttributeTok{col=}\StringTok{"orange"}\NormalTok{)}
\end{Highlighting}
\end{Shaded}

\includegraphics{Trabalho_files/figure-latex/unnamed-chunk-16-1.pdf}
Verificar a acurácia do modelo:

\begin{Shaded}
\begin{Highlighting}[]
\FunctionTok{accuracy}\NormalTok{(modelo\_tendencia\_poli\_proj, validacao\_ts)}
\end{Highlighting}
\end{Shaded}

\begin{verbatim}
##                         ME     RMSE       MAE        MPE     MAPE     MASE
## Training set  1.455458e-12 11273.48  9346.619  -2.920362 14.51938 2.999076
## Test set     -6.403989e+03 15167.09 14546.239 -12.235160 20.11448 4.667492
##                    ACF1 Theil's U
## Training set -0.1307409        NA
## Test set      0.1444087 0.9171101
\end{verbatim}

Calular o modelo ingenuo e verifica a sua acurácia:

\begin{Shaded}
\begin{Highlighting}[]
\NormalTok{modelo\_ingenuo }\OtherTok{\textless{}{-}} \FunctionTok{naive}\NormalTok{(treinamento\_ts, }\AttributeTok{level=}\DecValTok{0}\NormalTok{, }\AttributeTok{h=}\NormalTok{tam\_amostra\_teste)}
\FunctionTok{accuracy}\NormalTok{(modelo\_ingenuo, validacao\_ts)}
\end{Highlighting}
\end{Shaded}

\begin{verbatim}
##                      ME     RMSE      MAE         MPE     MAPE     MASE
## Training set   2212.421 17122.92 14065.58  -0.3423223 22.56419 4.513261
## Test set     -15674.500 22826.96 21251.50 -25.6460832 31.04299 6.819028
##                    ACF1 Theil's U
## Training set -0.2899524        NA
## Test set      0.1722954  1.331418
\end{verbatim}

\begin{Shaded}
\begin{Highlighting}[]
\FunctionTok{plot}\NormalTok{(modelo\_ingenuo, }\AttributeTok{xlab=}\StringTok{"Tempo"}\NormalTok{, }\AttributeTok{ylab=}\StringTok{"Vendas"}\NormalTok{, }\AttributeTok{xaxt=}\StringTok{"n"}\NormalTok{ ,}\AttributeTok{main=}\StringTok{"Previsão do Modelo Naive"}\NormalTok{,  }\AttributeTok{ylim=}\FunctionTok{c}\NormalTok{(}\DecValTok{40000}\NormalTok{, }\DecValTok{110000}\NormalTok{), }\AttributeTok{xlim=}\FunctionTok{c}\NormalTok{(}\DecValTok{2005}\NormalTok{, }\DecValTok{2010}\NormalTok{), }\AttributeTok{bty=}\StringTok{"l"}\NormalTok{, }\AttributeTok{flty=}\DecValTok{2}\NormalTok{)}

\FunctionTok{axis}\NormalTok{(}\DecValTok{1}\NormalTok{, }\AttributeTok{at=}\FunctionTok{seq}\NormalTok{(}\DecValTok{2005}\NormalTok{, }\DecValTok{2010}\NormalTok{, }\DecValTok{1}\NormalTok{), }\AttributeTok{labels=}\FunctionTok{format}\NormalTok{(}\FunctionTok{seq}\NormalTok{(}\DecValTok{2005}\NormalTok{, }\DecValTok{2010}\NormalTok{,}\DecValTok{1}\NormalTok{)))}

\FunctionTok{lines}\NormalTok{(validacao\_ts)}
\end{Highlighting}
\end{Shaded}

\includegraphics{Trabalho_files/figure-latex/unnamed-chunk-19-1.pdf}
Projetar o futuro

\begin{Shaded}
\begin{Highlighting}[]
\NormalTok{modelo\_tendencia\_poli\_final }\OtherTok{\textless{}{-}} \FunctionTok{tslm}\NormalTok{(sales\_ts }\SpecialCharTok{\textasciitilde{}}\NormalTok{ trend }\SpecialCharTok{+} \FunctionTok{I}\NormalTok{(trend}\SpecialCharTok{\^{}}\DecValTok{2}\NormalTok{))}
\FunctionTok{summary}\NormalTok{(modelo\_tendencia\_poli\_final)}
\end{Highlighting}
\end{Shaded}

\begin{verbatim}
## 
## Call:
## tslm(formula = sales_ts ~ trend + I(trend^2))
## 
## Residuals:
##    Min     1Q Median     3Q    Max 
## -15444  -8297  -4950   5517  20416 
## 
## Coefficients:
##             Estimate Std. Error t value Pr(>|t|)    
## (Intercept) 57028.00    8488.12   6.719  1.2e-06 ***
## trend        -420.03    1564.42  -0.268    0.791    
## I(trend^2)     63.57      60.75   1.046    0.307    
## ---
## Signif. codes:  0 '***' 0.001 '**' 0.01 '*' 0.05 '.' 0.1 ' ' 1
## 
## Residual standard error: 12720 on 21 degrees of freedom
## Multiple R-squared:  0.3398, Adjusted R-squared:  0.277 
## F-statistic: 5.405 on 2 and 21 DF,  p-value: 0.01277
\end{verbatim}

\begin{Shaded}
\begin{Highlighting}[]
\NormalTok{modelo\_tendencia\_poli\_final\_proj }\OtherTok{\textless{}{-}} \FunctionTok{forecast}\NormalTok{(modelo\_tendencia\_poli\_final, }\AttributeTok{h=}\DecValTok{36}\NormalTok{, }\AttributeTok{level=}\FloatTok{0.95}\NormalTok{)}
\end{Highlighting}
\end{Shaded}

PLotagem do futuro

\begin{Shaded}
\begin{Highlighting}[]
\FunctionTok{plot}\NormalTok{(modelo\_tendencia\_poli\_final\_proj, }\AttributeTok{xlab=}\StringTok{"Tempo"}\NormalTok{, }\AttributeTok{ylab=}\StringTok{"Vendas"}\NormalTok{, }\AttributeTok{ylim=}\FunctionTok{c}\NormalTok{(}\DecValTok{40000}\NormalTok{, }\DecValTok{110000}\NormalTok{), }\AttributeTok{xlim=}\FunctionTok{c}\NormalTok{(}\DecValTok{2005}\NormalTok{, }\DecValTok{2010}\NormalTok{), }\AttributeTok{bty=}\StringTok{"l"}\NormalTok{, }\AttributeTok{flty=}\DecValTok{2}\NormalTok{, }\AttributeTok{main=}\StringTok{"Forecast from Polynomial regression model"}\NormalTok{)}
\FunctionTok{axis}\NormalTok{(}\DecValTok{1}\NormalTok{, }\AttributeTok{at=}\FunctionTok{seq}\NormalTok{(}\DecValTok{2005}\NormalTok{, }\DecValTok{2010}\NormalTok{, }\DecValTok{1}\NormalTok{), }\AttributeTok{labels=}\FunctionTok{format}\NormalTok{(}\FunctionTok{seq}\NormalTok{(}\DecValTok{2005}\NormalTok{, }\DecValTok{2010}\NormalTok{,}\DecValTok{1}\NormalTok{)))}
\FunctionTok{lines}\NormalTok{(modelo\_tendencia\_poli\_final\_proj}\SpecialCharTok{$}\NormalTok{fitted, }\AttributeTok{lwd=}\DecValTok{2}\NormalTok{, }\AttributeTok{col=}\StringTok{"orange"}\NormalTok{)}
\end{Highlighting}
\end{Shaded}

\includegraphics{Trabalho_files/figure-latex/unnamed-chunk-21-1.pdf} 7.
Estimar um modelo de tendência linear com sazonalidade

\begin{Shaded}
\begin{Highlighting}[]
\FunctionTok{ggseasonplot}\NormalTok{(sales\_ts, }\AttributeTok{main=}\StringTok{"Vendas por quarto"}\NormalTok{)}
\end{Highlighting}
\end{Shaded}

\includegraphics{Trabalho_files/figure-latex/unnamed-chunk-22-1.pdf}

\begin{Shaded}
\begin{Highlighting}[]
\NormalTok{dummies\_mensais }\OtherTok{\textless{}{-}} \FunctionTok{seasonaldummy}\NormalTok{(sales\_ts)}
\NormalTok{modelo\_sazonalidade\_linear }\OtherTok{\textless{}{-}} \FunctionTok{tslm}\NormalTok{(treinamento\_ts }\SpecialCharTok{\textasciitilde{}}\NormalTok{ season)}
\FunctionTok{summary}\NormalTok{(modelo\_sazonalidade\_linear)}
\end{Highlighting}
\end{Shaded}

\begin{verbatim}
## 
## Call:
## tslm(formula = treinamento_ts ~ season)
## 
## Residuals:
##    Min     1Q Median     3Q    Max 
##  -5517  -3550  -1030   2999   9885 
## 
## Coefficients:
##             Estimate Std. Error t value Pr(>|t|)    
## (Intercept)    51534       2281  22.591 1.45e-13 ***
## season2         1946       3226   0.603  0.55483    
## season3        10782       3226   3.342  0.00414 ** 
## season4        30764       3226   9.536 5.30e-08 ***
## ---
## Signif. codes:  0 '***' 0.001 '**' 0.01 '*' 0.05 '.' 0.1 ' ' 1
## 
## Residual standard error: 5101 on 16 degrees of freedom
## Multiple R-squared:  0.877,  Adjusted R-squared:  0.8539 
## F-statistic: 38.02 on 3 and 16 DF,  p-value: 1.652e-07
\end{verbatim}

PLotando

\begin{Shaded}
\begin{Highlighting}[]
\FunctionTok{plot}\NormalTok{(modelo\_sazonalidade\_linear}\SpecialCharTok{$}\NormalTok{residuals, }\AttributeTok{xlab=}\StringTok{"Tempo"}\NormalTok{, }\AttributeTok{ylab=}\StringTok{"Resíduos"}\NormalTok{, }\AttributeTok{ylim=}\FunctionTok{c}\NormalTok{(}\SpecialCharTok{{-}}\DecValTok{10000}\NormalTok{, }\DecValTok{10000}\NormalTok{), }\AttributeTok{bty=}\StringTok{"l"}\NormalTok{)}
\end{Highlighting}
\end{Shaded}

\includegraphics{Trabalho_files/figure-latex/unnamed-chunk-23-1.pdf}
Calcula a Autocorrelação dos resíduos

\begin{Shaded}
\begin{Highlighting}[]
\FunctionTok{Acf}\NormalTok{(modelo\_sazonalidade\_linear}\SpecialCharTok{$}\NormalTok{residuals)}
\end{Highlighting}
\end{Shaded}

\includegraphics{Trabalho_files/figure-latex/unnamed-chunk-24-1.pdf}
Checagem dos resíduos com teste de Ljung - Box

\begin{Shaded}
\begin{Highlighting}[]
\FunctionTok{checkresiduals}\NormalTok{(modelo\_sazonalidade\_linear, }\AttributeTok{test=}\StringTok{"LB"}\NormalTok{)}
\end{Highlighting}
\end{Shaded}

\includegraphics{Trabalho_files/figure-latex/unnamed-chunk-25-1.pdf}

\begin{verbatim}
## 
##  Ljung-Box test
## 
## data:  Residuals from Linear regression model
## Q* = 31.276, df = 3, p-value = 7.437e-07
## 
## Model df: 4.   Total lags used: 7
\end{verbatim}

PLota o modelo com sazonalidade

\begin{Shaded}
\begin{Highlighting}[]
\FunctionTok{plot}\NormalTok{(treinamento\_ts, }\AttributeTok{xlab=}\StringTok{"Tempo"}\NormalTok{, }\AttributeTok{ylab=}\StringTok{"Vendas"}\NormalTok{, }\AttributeTok{ylim=}\FunctionTok{c}\NormalTok{(}\DecValTok{50000}\NormalTok{, }\DecValTok{110000}\NormalTok{), }\AttributeTok{bty=}\StringTok{"l"}\NormalTok{)}
\FunctionTok{lines}\NormalTok{(modelo\_sazonalidade\_linear}\SpecialCharTok{$}\NormalTok{fitted.values, }\AttributeTok{lwd=}\DecValTok{2}\NormalTok{, }\AttributeTok{col=}\StringTok{"blue"}\NormalTok{)}
\end{Highlighting}
\end{Shaded}

\includegraphics{Trabalho_files/figure-latex/unnamed-chunk-26-1.pdf}

\end{document}
